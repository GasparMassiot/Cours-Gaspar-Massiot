% Options for packages loaded elsewhere
\PassOptionsToPackage{unicode}{hyperref}
\PassOptionsToPackage{hyphens}{url}
%
\documentclass[
]{article}
\usepackage{amsmath,amssymb}
\usepackage{iftex}
\ifPDFTeX
  \usepackage[T1]{fontenc}
  \usepackage[utf8]{inputenc}
  \usepackage{textcomp} % provide euro and other symbols
\else % if luatex or xetex
  \usepackage{unicode-math} % this also loads fontspec
  \defaultfontfeatures{Scale=MatchLowercase}
  \defaultfontfeatures[\rmfamily]{Ligatures=TeX,Scale=1}
\fi
\usepackage{lmodern}
\ifPDFTeX\else
  % xetex/luatex font selection
\fi
% Use upquote if available, for straight quotes in verbatim environments
\IfFileExists{upquote.sty}{\usepackage{upquote}}{}
\IfFileExists{microtype.sty}{% use microtype if available
  \usepackage[]{microtype}
  \UseMicrotypeSet[protrusion]{basicmath} % disable protrusion for tt fonts
}{}
\makeatletter
\@ifundefined{KOMAClassName}{% if non-KOMA class
  \IfFileExists{parskip.sty}{%
    \usepackage{parskip}
  }{% else
    \setlength{\parindent}{0pt}
    \setlength{\parskip}{6pt plus 2pt minus 1pt}}
}{% if KOMA class
  \KOMAoptions{parskip=half}}
\makeatother
\usepackage{xcolor}
\usepackage[margin=1in]{geometry}
\usepackage{color}
\usepackage{fancyvrb}
\newcommand{\VerbBar}{|}
\newcommand{\VERB}{\Verb[commandchars=\\\{\}]}
\DefineVerbatimEnvironment{Highlighting}{Verbatim}{commandchars=\\\{\}}
% Add ',fontsize=\small' for more characters per line
\usepackage{framed}
\definecolor{shadecolor}{RGB}{248,248,248}
\newenvironment{Shaded}{\begin{snugshade}}{\end{snugshade}}
\newcommand{\AlertTok}[1]{\textcolor[rgb]{0.94,0.16,0.16}{#1}}
\newcommand{\AnnotationTok}[1]{\textcolor[rgb]{0.56,0.35,0.01}{\textbf{\textit{#1}}}}
\newcommand{\AttributeTok}[1]{\textcolor[rgb]{0.13,0.29,0.53}{#1}}
\newcommand{\BaseNTok}[1]{\textcolor[rgb]{0.00,0.00,0.81}{#1}}
\newcommand{\BuiltInTok}[1]{#1}
\newcommand{\CharTok}[1]{\textcolor[rgb]{0.31,0.60,0.02}{#1}}
\newcommand{\CommentTok}[1]{\textcolor[rgb]{0.56,0.35,0.01}{\textit{#1}}}
\newcommand{\CommentVarTok}[1]{\textcolor[rgb]{0.56,0.35,0.01}{\textbf{\textit{#1}}}}
\newcommand{\ConstantTok}[1]{\textcolor[rgb]{0.56,0.35,0.01}{#1}}
\newcommand{\ControlFlowTok}[1]{\textcolor[rgb]{0.13,0.29,0.53}{\textbf{#1}}}
\newcommand{\DataTypeTok}[1]{\textcolor[rgb]{0.13,0.29,0.53}{#1}}
\newcommand{\DecValTok}[1]{\textcolor[rgb]{0.00,0.00,0.81}{#1}}
\newcommand{\DocumentationTok}[1]{\textcolor[rgb]{0.56,0.35,0.01}{\textbf{\textit{#1}}}}
\newcommand{\ErrorTok}[1]{\textcolor[rgb]{0.64,0.00,0.00}{\textbf{#1}}}
\newcommand{\ExtensionTok}[1]{#1}
\newcommand{\FloatTok}[1]{\textcolor[rgb]{0.00,0.00,0.81}{#1}}
\newcommand{\FunctionTok}[1]{\textcolor[rgb]{0.13,0.29,0.53}{\textbf{#1}}}
\newcommand{\ImportTok}[1]{#1}
\newcommand{\InformationTok}[1]{\textcolor[rgb]{0.56,0.35,0.01}{\textbf{\textit{#1}}}}
\newcommand{\KeywordTok}[1]{\textcolor[rgb]{0.13,0.29,0.53}{\textbf{#1}}}
\newcommand{\NormalTok}[1]{#1}
\newcommand{\OperatorTok}[1]{\textcolor[rgb]{0.81,0.36,0.00}{\textbf{#1}}}
\newcommand{\OtherTok}[1]{\textcolor[rgb]{0.56,0.35,0.01}{#1}}
\newcommand{\PreprocessorTok}[1]{\textcolor[rgb]{0.56,0.35,0.01}{\textit{#1}}}
\newcommand{\RegionMarkerTok}[1]{#1}
\newcommand{\SpecialCharTok}[1]{\textcolor[rgb]{0.81,0.36,0.00}{\textbf{#1}}}
\newcommand{\SpecialStringTok}[1]{\textcolor[rgb]{0.31,0.60,0.02}{#1}}
\newcommand{\StringTok}[1]{\textcolor[rgb]{0.31,0.60,0.02}{#1}}
\newcommand{\VariableTok}[1]{\textcolor[rgb]{0.00,0.00,0.00}{#1}}
\newcommand{\VerbatimStringTok}[1]{\textcolor[rgb]{0.31,0.60,0.02}{#1}}
\newcommand{\WarningTok}[1]{\textcolor[rgb]{0.56,0.35,0.01}{\textbf{\textit{#1}}}}
\usepackage{longtable,booktabs,array}
\usepackage{calc} % for calculating minipage widths
% Correct order of tables after \paragraph or \subparagraph
\usepackage{etoolbox}
\makeatletter
\patchcmd\longtable{\par}{\if@noskipsec\mbox{}\fi\par}{}{}
\makeatother
% Allow footnotes in longtable head/foot
\IfFileExists{footnotehyper.sty}{\usepackage{footnotehyper}}{\usepackage{footnote}}
\makesavenoteenv{longtable}
\usepackage{graphicx}
\makeatletter
\def\maxwidth{\ifdim\Gin@nat@width>\linewidth\linewidth\else\Gin@nat@width\fi}
\def\maxheight{\ifdim\Gin@nat@height>\textheight\textheight\else\Gin@nat@height\fi}
\makeatother
% Scale images if necessary, so that they will not overflow the page
% margins by default, and it is still possible to overwrite the defaults
% using explicit options in \includegraphics[width, height, ...]{}
\setkeys{Gin}{width=\maxwidth,height=\maxheight,keepaspectratio}
% Set default figure placement to htbp
\makeatletter
\def\fps@figure{htbp}
\makeatother
\setlength{\emergencystretch}{3em} % prevent overfull lines
\providecommand{\tightlist}{%
  \setlength{\itemsep}{0pt}\setlength{\parskip}{0pt}}
\setcounter{secnumdepth}{-\maxdimen} % remove section numbering
\ifLuaTeX
  \usepackage{selnolig}  % disable illegal ligatures
\fi
\usepackage{bookmark}
\IfFileExists{xurl.sty}{\usepackage{xurl}}{} % add URL line breaks if available
\urlstyle{same}
\hypersetup{
  pdftitle={Analyse de données Travaux Dirigés et Pratiques},
  pdfauthor={Fiche 1 : Rappels de Statistiques Descriptives},
  hidelinks,
  pdfcreator={LaTeX via pandoc}}

\title{Analyse de données Travaux Dirigés et Pratiques}
\author{Fiche 1 : Rappels de Statistiques Descriptives}
\date{}

\begin{document}
\maketitle

\subsection{Exercice 1}\label{exercice-1}

\begin{enumerate}
\def\labelenumi{\arabic{enumi}.}
\tightlist
\item
  Pour chacune des variables suivantes, préciser son type :
\end{enumerate}

\begin{minipage}[t]{0.5\textwidth}
\begin{itemize}
\item Revenu annuel           
\item Sexe                    
\item Etat matrimonial        
\item Lieu de résidence       
\item Pointure de chaussures 
\end{itemize}
\end{minipage}
\begin{minipage}[t]{0.5\textwidth}
\begin{itemize}
\item Citoyenneté
\item Couleur des yeux
\item Nombre de langues parlées
\item Âge
\item Tour de taille
\end{itemize}
\end{minipage}

\begin{enumerate}
\def\labelenumi{\arabic{enumi}.}
\setcounter{enumi}{1}
\item
  Rappeler quelques mesures de la tendance centrale et leurs principales
  caractéristiques (qualités et défauts).
\item
  Même question pour des mesures de la dispersion.
\item
  Lorsque la distribution est symétrique, que peut-on dire de la
  moyenne, la médiane et le mode ?
\item
  Quelle est la différence entre valeurs manquantes, valeurs aberrantes
  et valeurs extrêmes ?
\end{enumerate}

\subsection{Exercice 2 Nettoyage et prétraitement des
données}\label{exercice-2-nettoyage-et-pruxe9traitement-des-donnuxe9es}

\emph{Cet exercice sera effectué sous le logiciel R.}

\begin{enumerate}
\def\labelenumi{\arabic{enumi}.}
\item
  Créer un dossier \texttt{TP\_StatDes} dans votre espace personnel.
  Créer un sous-dossier \texttt{Données} et un sous dossier
  \texttt{Script} dans ce dossier.
\item
  Télécharger les données \texttt{Recensement\_12.csv} sur Moodle ou sur
  le répertoire \texttt{COMMUN}. Ces données sont un échantillon de 599
  foyers du recensement effectué en 2012 aux Etats Unis, décrits par 11
  variables. Enregistrer ces données dans le sous dossier
  \texttt{données} du dossier \texttt{TP\_StatDes}.
\item
  Ouvrir \texttt{RStudio} et créer un nouveau projet :
  \texttt{File\ \textgreater{}\ New\ Project...}. Choisir
  \texttt{Existing\ Directory} et sélectionner le dossier
  \texttt{TP\_StatDes}.

  \includegraphics{Creer_projet.png}
\item
  Dans \texttt{RStudio}, créer un nouveau script \texttt{R} nommé
  \texttt{TP1} :
  \texttt{File\ \textgreater{}\ New\ File\ \textgreater{}\ R\ Script}
  (raccourci clavier \texttt{Ctrl+Maj+N}) et l'enregistrer dans le sous
  dossier \texttt{script} du dossier \texttt{TP\_StatDes}.
\item
  Charger les données à l'aide de la commande suivante : à quoi servent
  les options \texttt{row.names\ =\ 1} et
  \texttt{stringsAsFactors\ =\ T} ?
\end{enumerate}

\emph{Bien qu'en anglais, l'aide de \texttt{R} est relativement bien
conçue. Il est souvent utile de s'y référer pour comprendre
l'utilisation d'une fonction donnée. On pourra par exemple lancer la
commande \texttt{help(read.csv2)}, pour avoir plus de détails sur cette
fonction. Le raccourcis clavier pour accéder à l'aider est \texttt{F1}.}

\begin{Shaded}
\begin{Highlighting}[]
\CommentTok{\# Chargement des données}
\NormalTok{Recensement\_12 }\OtherTok{\textless{}{-}} \FunctionTok{read.csv2}\NormalTok{(}\StringTok{"Données/Recensement\_12.csv"}\NormalTok{, }\AttributeTok{row.names =} \DecValTok{1}\NormalTok{, }\AttributeTok{stringsAsFactors =}\NormalTok{ T)}
\end{Highlighting}
\end{Shaded}

\begin{enumerate}
\def\labelenumi{\arabic{enumi}.}
\setcounter{enumi}{4}
\tightlist
\item
  A l'aide des fonctions suivantes, donner une description rapide des
  données; Existe-t-il des valeurs manquantes ou aberrantes ?
\end{enumerate}

\begin{Shaded}
\begin{Highlighting}[]
\CommentTok{\# fonction str, permet d\textquotesingle{}avoir un aperçu du type des variables}
\FunctionTok{str}\NormalTok{(Recensement\_12)}
\CommentTok{\# fontion summary de base}
\FunctionTok{summary}\NormalTok{(Recensement\_12)}
\CommentTok{\# fonction skim, alternative à summary et str}
\NormalTok{skimr}\SpecialCharTok{::}\FunctionTok{skim}\NormalTok{(Recensement\_12)}
\CommentTok{\# fonction glimpse du package dplyr}
\NormalTok{dplyr}\SpecialCharTok{::}\FunctionTok{glimpse}\NormalTok{(Recensement\_12)}
\end{Highlighting}
\end{Shaded}

\emph{Il est possible que les package \texttt{skimr} et \texttt{dplyr}
ne soient pas installés sur vos machines, dans ce cas les appels
\texttt{install.packages("skimr")} et \texttt{install.packages("dplyr")}
les installeront. Une fois installés, on peut charger les package à
l'aide des invites \texttt{library(skimr)} et \texttt{library(dplyr)},
cela permet d'éviter de répéter les \texttt{dplyr::} et \texttt{skimr::}
dans le code précédent.}

\begin{enumerate}
\def\labelenumi{\arabic{enumi}.}
\setcounter{enumi}{5}
\item
  Calculer le pourcentage de valeurs manquantes dans la table. On pourra
  par exemple utiliser les fonctions de base de \texttt{R} suivantes :
  \texttt{is.na}, \texttt{sum}, \texttt{nrow} et \texttt{ncol}.
\item
  On souhaite maintenant éliminer les variables ayant plus de 70\% de
  valeurs manquantes et dans un deuxième temps retirer les individus
  ayant plus de 60\% de valeurs manquantes. Cela vous semble-t-il
  judicieux pour l'analyse ? Quelles sont les conséquences d'une telle
  opération ? Pour coder cette opération on peut utiliser les fonctions
  de base de \texttt{R} telles que \texttt{colSums} et \texttt{rowSums},
  on peut aussi utiliser les fonctions du \texttt{tidyverse} tel que
  présenté ci-après.
\end{enumerate}

\begin{Shaded}
\begin{Highlighting}[]
\NormalTok{Recensement\_12 }\OtherTok{\textless{}{-}}\NormalTok{ Recensement\_12 }\SpecialCharTok{\%\textgreater{}\%} 
  \DocumentationTok{\#\#\# select permet de sélectionner des variables}
  \DocumentationTok{\#\#\# where permet d\textquotesingle{}appliquer une condition définie par une formule}
  \FunctionTok{select}\NormalTok{(}\FunctionTok{where}\NormalTok{(}\SpecialCharTok{\textasciitilde{}} \FunctionTok{sum}\NormalTok{(}\FunctionTok{is.na}\NormalTok{(.x))}\SpecialCharTok{/}\FunctionTok{length}\NormalTok{(.x)}\SpecialCharTok{\textless{}}\FloatTok{0.7}\NormalTok{))}

\NormalTok{Recensement\_12 }\OtherTok{\textless{}{-}}\NormalTok{ Recensement\_12 }\SpecialCharTok{\%\textgreater{}\%} 
  \CommentTok{\# Calcul du pourcentage de NA par individu}
  \FunctionTok{mutate}\NormalTok{(}\AttributeTok{percentage\_NA =} \FunctionTok{rowSums}\NormalTok{(}\FunctionTok{is.na}\NormalTok{(Recensement\_12))}\SpecialCharTok{/}\FunctionTok{ncol}\NormalTok{(Recensement\_12)) }\SpecialCharTok{\%\textgreater{}\%}
  \CommentTok{\# Sélection des individus selon la condition}
  \FunctionTok{filter}\NormalTok{(percentage\_NA}\SpecialCharTok{\textless{}}\FloatTok{0.6}\NormalTok{) }\SpecialCharTok{\%\textgreater{}\%}
  \CommentTok{\# Suppression de la variable percentage\_NA}
  \FunctionTok{select}\NormalTok{(}\SpecialCharTok{{-}}\NormalTok{percentage\_NA)}
\end{Highlighting}
\end{Shaded}

\begin{enumerate}
\def\labelenumi{\arabic{enumi}.}
\setcounter{enumi}{7}
\item
  Séparer la table de données obtenue en 2 tables

  \begin{itemize}
  \tightlist
  \item
    une table \texttt{quali} contenant toutes les variables qualitatives
  \item
    une table \texttt{quanti} contenant toutes les variables
    quantitatives.
  \end{itemize}
\end{enumerate}

\emph{On pourra s'inspirer du code précédant et utiliser les fonctions
\texttt{select} et \texttt{where} du \texttt{tidyverse} ainsi que des
fonctions \texttt{is.numeric} et \texttt{is.factor} de \texttt{Rbase}.}

\begin{enumerate}
\def\labelenumi{\arabic{enumi}.}
\setcounter{enumi}{9}
\item
  On souhaite imputer les données manquantes dans la base
  \texttt{quanti}, quelle(s) méthode(s) connaissez-vous pour ce faire ?
\item
  Existe-t-il un moyen de faire la même chose pour les données
  manquantes de la base \texttt{quali} ?
\item
  La fonction \texttt{na.aggregate} du package \texttt{zoo} permet
  d'imputer les données (e.g.~moyenne ou médiane). Utilisez la pour
  créer la table \texttt{dat} à partir de la table \texttt{quanti} et
  dans laquelle les valeurs manquantes ont été imputées par la médiane
  pour chaque variable.
\end{enumerate}

\subsection{Exercice 3 Statistiques descriptives
univariées}\label{exercice-3-statistiques-descriptives-univariuxe9es}

\emph{Cet exercice sera effectué sous le logiciel R}

\begin{enumerate}
\def\labelenumi{\arabic{enumi}.}
\item
  \textbf{Variables qualitatives}

  \begin{enumerate}
  \def\labelenumii{\alph{enumii}.}
  \item
    Pour chaque variable du jeu de données \texttt{quali}, choisir la
    représentation la plus adaptée. On pourra s'inspirer du site
    \href{https://r-graph-gallery.com/}{R Graph Gallery}.
  \item
    On rappelle que la commande \texttt{prop.table(table(x))} permet de
    calculer les fréquences de chaque modalité du vecteur \texttt{x} et
    que les fonctions \texttt{pie} et \texttt{barplot} permettent de
    construire respectivement un diagramme circulaire et un diagramme en
    barres. Les fonctions \texttt{geom\_bar} et \texttt{geom\_col} du
    package \texttt{ggplot2} permmettent également de construire de tels
    graphiques.
  \item
    Interpréter les graphiques obtenus.
  \end{enumerate}
\item
  \textbf{Variables quantitatives}

  \begin{enumerate}
  \def\labelenumii{\alph{enumii}.}
  \tightlist
  \item
    Interpréter les sorties des commandes suivantes
  \end{enumerate}

\begin{Shaded}
\begin{Highlighting}[]
\CommentTok{\# Fonction summary}
\FunctionTok{summary}\NormalTok{(dat)}

\CommentTok{\# Fonction skim du package skimr}
\NormalTok{skimr}\SpecialCharTok{::}\FunctionTok{skim}\NormalTok{(dat)}
\end{Highlighting}
\end{Shaded}

  \begin{enumerate}
  \def\labelenumii{\alph{enumii}.}
  \setcounter{enumii}{1}
  \tightlist
  \item
    Taper les commandes
  \end{enumerate}

\begin{Shaded}
\begin{Highlighting}[]
\FunctionTok{var}\NormalTok{(dat[,}\DecValTok{1}\NormalTok{])}
\FunctionTok{sum}\NormalTok{((dat[,}\DecValTok{1}\NormalTok{]}\SpecialCharTok{{-}}\FunctionTok{mean}\NormalTok{(dat[,}\DecValTok{1}\NormalTok{]))}\SpecialCharTok{\^{}}\DecValTok{2}\SpecialCharTok{/}\FunctionTok{nrow}\NormalTok{(dat))}
\end{Highlighting}
\end{Shaded}

  Que remarquez vous ? Utiliser l'aide de la fonction \texttt{var} afin
  de trouver une explication et adapter la deuxième ligne le cas
  échéant.

  \begin{enumerate}
  \def\labelenumii{\alph{enumii}.}
  \setcounter{enumii}{2}
  \item
    Pour chaque variable quantitative, choisir la représentation la plus
    adaptée. On pourra s'inspirer du site
    \href{https://r-graph-gallery.com/}{R Graph Gallery}.
  \item
    Suivant le type de graphique choisi, on pourra utiliser les
    fonctions suivantes :
  \end{enumerate}

  \begin{longtable}[]{@{}
    >{\centering\arraybackslash}p{(\columnwidth - 4\tabcolsep) * \real{0.3833}}
    >{\centering\arraybackslash}p{(\columnwidth - 4\tabcolsep) * \real{0.2500}}
    >{\centering\arraybackslash}p{(\columnwidth - 4\tabcolsep) * \real{0.3667}}@{}}
  \toprule\noalign{}
  \begin{minipage}[b]{\linewidth}\centering
  Diagramme en barres
  \end{minipage} & \begin{minipage}[b]{\linewidth}\centering
  Histogramme
  \end{minipage} & \begin{minipage}[b]{\linewidth}\centering
  Boîte à moustaches
  \end{minipage} \\
  \midrule\noalign{}
  \endhead
  \bottomrule\noalign{}
  \endlastfoot
  \texttt{barplot} & \texttt{hist} & \texttt{boxplot} \\
  \texttt{geom\_bar} ou \texttt{geom\_col} & \texttt{geom\_histogram} ou
  \texttt{geom\_density} & \texttt{geom\_boxplot} \\
  \end{longtable}

  \begin{enumerate}
  \def\labelenumii{\alph{enumii}.}
  \setcounter{enumii}{4}
  \tightlist
  \item
    Interpréter les graphiques obtenus
  \end{enumerate}
\end{enumerate}

\subsection{Exercice 4 Statistiques descriptives
bivariées}\label{exercice-4-statistiques-descriptives-bivariuxe9es}

\emph{Cet exercice sera effectué sous le logiciel R.}

\begin{enumerate}
\def\labelenumi{\arabic{enumi}.}
\item
  Indiquer les différents types d'outils statistiques utilisés pour
  interpréter les liens éventuels entre 2 variables
\item
  Parmi les variables du jeu de données \texttt{Recensement\_12},
  sélectionner plusieurs couples de variables et utiliser les outils
  statistiques adéquats afin d'interpréter leurs relations. Les
  commandes suivantes peuvent être utiles
\end{enumerate}

\begin{longtable}[]{@{}
  >{\centering\arraybackslash}p{(\columnwidth - 6\tabcolsep) * \real{0.3014}}
  >{\centering\arraybackslash}p{(\columnwidth - 6\tabcolsep) * \real{0.1781}}
  >{\centering\arraybackslash}p{(\columnwidth - 6\tabcolsep) * \real{0.2740}}
  >{\centering\arraybackslash}p{(\columnwidth - 6\tabcolsep) * \real{0.2466}}@{}}
\toprule\noalign{}
\begin{minipage}[b]{\linewidth}\centering
Table de contingence
\end{minipage} & \begin{minipage}[b]{\linewidth}\centering
Corrélation
\end{minipage} & \begin{minipage}[b]{\linewidth}\centering
Boîte à moustaches
\end{minipage} & \begin{minipage}[b]{\linewidth}\centering
Nuages de points
\end{minipage} \\
\midrule\noalign{}
\endhead
\bottomrule\noalign{}
\endlastfoot
\texttt{table(x,y)} & \texttt{cov(dat)}, \texttt{cor(dat)} &
\texttt{boxplot(x\ \textasciitilde{}\ y)} & \texttt{plot(x,y)},
\texttt{pairs(dat)} \\
\texttt{dat\ \%\textgreater{}\%\ count(x,y)} &
\texttt{corrplot(cor(dat))} &
\texttt{ggplot(dat)\ +\ geom\_boxplot(aes(x,\ group\ =\ y))} &
\texttt{ggplot(dat)\ +\ geom\_point(aes(x,\ y))} \\
\end{longtable}

\end{document}
